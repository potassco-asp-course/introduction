% ----------------------------------------------------------------------
\begin{frame}{What is the meaning of a logic program?}

  \begin{itemize}
  \item<only@2-12> \structure{Lessons learned} from positive programs
    \begin{itemize}\normalsize
    \item \Cn{P} is the smallest \alt<6-11>{model of}{set closed under} $P$, eliminating all others
    \item Every atom in \Cn{P} is justified by a proof
      \medskip
    \item<only@3-4> \structure{Example} \
      $P=\{a\leftarrow{},b\leftarrow a, d\leftarrow c\}$
      \ yields \
      $\{a,b\}$
      \ only
      \begin{itemize}\normalsize
      \item<4> $a$ \ is justified by \ $a\leftarrow$
      \item<4> $b$ \ is justified by \ $b\leftarrow a$ and $a\leftarrow{}$
      \end{itemize}
    \end{itemize}

  \item<only@6-11> \structure{Logical attempt}\only<11>{ --- \textcolor{red}{failed}}
    \begin{itemize}\normalsize
    \item<7-> \structure{Hypothesis} \ The smallest model of $P$\,?
    \item<only@8-> \structure{Example} \ $P=\{a\leftarrow\neg b\}$ \ yields (stable model)\ $\{a\}$
    \item<only@9->[] but $P$ has \alt<10->{two minimal}{three} models, $\{a\}$\alt<10->{ and}{,} $\{b\}$
      \only<9>{, and $\{a,b\}$}
    \end{itemize}

  \item<only@12-> \structure{Procedural attempt\only<39->{ patched}}
    \only<38>{ --- \ \textcolor{red}{failed}}%
    \only<68>{ --- \ \textcolor{yellow}{interesting \dots}}
  \item<only@39-> [] \textbf{guess} $Y$
  \item<only@13-> [] \textbf{let} $\T{P}{X}=\{\head{r}\mid r\in P, \pbody{r}\subseteq X\only<14->{,\nbody{r}\cap \alt<39->{Y}{X}=\emptyset}\}$ \textbf{in}
    \begin{itemize}\normalsize
    \item []$X := \emptyset$
      \smallskip
    \item []\textbf{while} {$\T{P}{X}\neq X$}
      \begin{itemize}\normalsize
      \item[] $X := {\,\T{P}{X}}$
      \end{itemize}
    \item[] \only<39->{\textbf{if} $X=Y$ \textbf{then} }\textbf{return} $X$\only<39->{ \textbf{else} \textbf{fail}}
    \end{itemize}
    \medskip
  \item<only@15-37,40-67> \structure{Example} \
    $P=\{a\leftarrow{},\ b\leftarrow a,\neg c\only<24-37,55->{,\ c\leftarrow b}\}$
    \qquad
    \only<41-54,56->{$Y=\alt<62->{\{a,b,c\}}{\alt<52-54,59-61>{\{a,b\}}{\alt<49-51>{\{b\}}{\alt<46-48,56-58>{\{a\}}{\emptyset}}}}$}
    \only<45,48,51,58,61,67>{\ \KO}\only<54>{\ \OK}

    \medskip

    \begin{tabular}{ccccc}
      \only<16-22,25-36,42-45,47-48,50-51,53-54,57-58,60-61,62->{\      $X_0=\emptyset$}&
      \only<18-22,27-36,42-45,47-48,50-51,53-54,57-58,60-61,64->{\      $X_1=\{a    \}$}&
      \only<20-22,29-36,42-45,47-48,50-51,53-54,57-58,60-61    >{\quad  $X_2=\{a,b  \}$}&
      \only<      31-36,                        57-58,60-61    >{\quad  $X_3=\{a,b,c\}$}&
      \only<22         ,43-45,47-48,50-51,53-54,            65->{         $X=\{a\only<-60>{,b}\}\only<44-45,47-48,50-51,66-67>{\neq Y}\only<53-54>{=Y}$}\\
      \only<17-22,26-36,42-45,47-48,50-51,53-54,57-58,60-61,63->{$\T{P}{X_0}=\{a\}$}&
      \only<19-22,28-36,42-45,47-48,50-51,53-54,57-58,60-61,65->{$\T{P}{X_1}=\{a\only<           -64>{,b}                            \}$}&
      \only<21-22,30-36,42-45,47-48,50-51,53-54,57-58,60-61    >{$\T{P}{X_2}=\{a                      ,b \only<30-37,57-58,60-61>{,c}\}$}&
      \only<      32-36,                        57-58,60-61    >{$\T{P}{X_3}=\{a\only<57-58,60-61   >{,b}\only<32-36,57-58,60-61>{,c}\}$}&
      \only<33-36>{\\[10pt]
      &\only<35-36>{$X_5=\{a\}$}&&\only<33-36>{      $X_4 =\{a,c\}$}&\\
      &\only<36-36>{   \dots   }&&\only<34-36>{$\T{P}{X_4}=\{a  \}$}&}
    \end{tabular}

    \item<only@69-> \structure{Question} \ Can this idea be used for a mathematical characterization\,?

      \only<70->{%
      \begin{enumerate}
      \item<70-> guess $Y$
      \item<71-> evaluate \nbody{r} of each rule $r\in P$ wrt to $Y$
        \begin{enumerate}
        \item<72-> if $\nbody{r}\cap Y\neq\emptyset$ drop $r$
        \item<72-> if $\nbody{r}\cap Y  = \emptyset$ replace \body{r} by \pbody{r} from $r$
        \end{enumerate}
      \item<73-> if $Y=\Cn{P'}$ for the resulting (positive) program $P'$ then \OK\ else \KO\
      \end{enumerate}}

  \end{itemize}
\end{frame}
% ----------------------------------------------------------------------
%
%%% Local Variables:
%%% mode: latex
%%% TeX-master: "../../main"
%%% End:
