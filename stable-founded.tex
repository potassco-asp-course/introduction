\newcommand{\STROUT}[2]{\alt<#1>{\text{\textcolor{red}{\sout{\ensuremath{\{#2\}}}}}}{\ensuremath{\{#2\}}}}
% ----------------------------------------------------------------------
\begin{frame}[c]{Intuition}
  \begin{itemize}
  \item A logic program is associated with a set of \alert{answer sets} induced
    \par by their applying rules
    \smallskip
  \item The concept of an answer set is fixed by a criterion rather than
    \par a construction process
    \medskip
  \item<2-> An answer set is an interpretation of all atoms with true or false
    \par such that
    \begin{itemize}\normalsize
    \item all rules are satisfied and
    \item all true atoms are founded
    \end{itemize}
    \medskip
  \item<3-> \structure{Note} \ Foundedness in database systems
\end{itemize}
\end{frame}
% ----------------------------------------------------------------------
\begin{frame}[c]{Interpretation}
  \begin{itemize}
  \item An \alert{interpretation} associates each atom with either the truth value \textit{true} or \textit{false}
  \item<2-> \structure{Example}
    \par
    \begin{tabular}{rcl}\normalsize
      \lstinline{answer(42)}            & $\mapsto$ & \textit{true}\\
      \lstinline{bachelor(friend(joe))} & $\mapsto$ & \textit{false}\\
      \lstinline{hot}                   & $\mapsto$ & \textit{false}\\
    \end{tabular}
  \item <3> \structure{Note} \ Compare with poetry and arithmetic, eg \ $\textrm{IX}\mapsto \textit{``}\;9\textit{''}$
  \item <4-> We represent an interpretation by the set of its true atoms
    \medskip
  \item <5-> \structure{Example} \
    \(
      \{\,\texttt{answer(42)}\,\}
    \)
  \item[]<5-> \structure{Note} \ \lstinline{bachelor(friend(joe))} and \lstinline{hot} are left out
  \end{itemize}
\end{frame}
% ----------------------------------------------------------------------
\begin{frame}[c]{Model}
  \begin{itemize}
  \item A \alert{model} of a logic program is an interpretation that satisfies
    \par each rule in the program
    \smallskip
  \item<2-> An interpretation $X$ \alert{satisfies} a rule
    \[
      {\tt a\texttt{ :- } b_1,\dots,b_m,\texttt{ not }{c_{m+1}},\dots,\texttt{ not }{c_n}.}
    \]
    whenever the following condition holds
    \[
      \mathtt{a}\in X \textbf{ if } \mathtt{b_1}\in X,\dots,\mathtt{b_m}\in X\textbf{ and }\mathtt{c_1}\notin X,\dots,\mathtt{c_n}\notin X
      % \hfill (\textnormal{SAT})\quad
    \]
  \end{itemize}
\end{frame}
% ----------------------------------------------------------------------
\begin{frame}[c]{Example}
  \begin{itemize}
  \item Consider the rule
    \[
      {\tt a\texttt{ :- } b,\texttt{ not }c.}
      \pause\quad\equiv\quad
      \mathtt{a}\in X \textbf{ if } \mathtt{b}\in X\textbf{ and }\mathtt{c}\notin X
    \]
  \item <3-> This rule has seven models
    \begin{align*}
      &\{\}\\
      &\{c\}\\
      &\STROUT{4}{b}\\
      &\{b,c\}\\
      &\{a\}\\
      &\{a,c\}\\
      &\{a,b\}\\
      &\{a,b,c\}
    \end{align*}
  \end{itemize}
\end{frame}
% ----------------------------------------------------------------------
\begin{frame}[c]{Example}
  \begin{itemize}
  \item Consider the rules
    \begin{align*}
      &{\tt b.}\\
      &{\tt a\texttt{ :- } b,\texttt{ not }c.}
    \end{align*}
  \item <2-> This rule has three models
    \begin{align*}
      &\STROUT{3-}{}\\
      &\STROUT{3-}{c}\\
      &\STROUT{4-}{b}\\
      &\{b,c\}\\
      &\STROUT{3-}{a}\\
      &\STROUT{3-}{a,c}\\
      &\{a,b\}\\
      &\{a,b,c\}
    \end{align*}
  \end{itemize}
\end{frame}
% ----------------------------------------------------------------------
\begin{frame}[c]{Example}
  \begin{itemize}
  \item Consider the rules
    \begin{align*}
      &{\tt b.\quad   c.}\\
      &{\tt a\texttt{ :- } b,\texttt{ not }c.}
    \end{align*}
  \item <2-> This rule has two models
    \begin{align*}
      &\STROUT{3-}{}\\
      &\STROUT{3-}{c}\\
      &\STROUT{3-}{b}\\
      &\{b,c\}\\
      &\STROUT{3-}{a}\\
      &\STROUT{3-}{a,c}\\
      &\STROUT{3-}{a,b}\\
      &\{a,b,c\}
    \end{align*}
  \end{itemize}
\end{frame}
% ----------------------------------------------------------------------
\begin{frame}[c]{Stable model}
  \begin{itemize}
  \item An interpretation is \alert{founded} by a logic program,
    \par\smallskip
    if it is the smallest set $X$ satisfying the following properties
    \smallskip
    \begin{enumerate}\normalsize
    \item $\texttt{a}\in X$\textbf{ if }`\texttt{a.}' is a fact
      \medskip
    \item $\texttt{a}\in X$\textbf{ if }`${\tt a\texttt{ :- } b_1,\dots,b_m,\,\texttt{not}\,c_{m+1},\dots,\,\texttt{not}\,c_n.}$' is a rule
      \par\smallskip
      such that
      \begin{enumerate}\normalsize
      \item $\mathtt{b_1},\dots,\mathtt{b_m}$ is founded and
      \item $\mathtt{c_1}\notin X,\dots,\mathtt{c_n}\notin X$
      \end{enumerate}
    \end{enumerate}
    \medskip
  \item<2-> \structure{Note} \ Foundedness relies foremost on atoms within $X$

  \end{itemize}
\end{frame}
% ----------------------------------------------------------------------
\begin{frame}[c]{Example}
  \begin{itemize}
  \item Consider the rule
    \[
      {\tt a\texttt{ :- } b,\texttt{ not }c.}
    \]
  \item <3-> This rule has one stable models
    \begin{align*}
      &\{\}\\
      &\STROUT{4-}{c}\\
      &\STROUT{3-}{b}\\
      &\STROUT{4-}{b,c}\\
      &\STROUT{5-}{a}\\
      &\STROUT{4-}{a,c}\\
      &\STROUT{4-}{a,b}\\
      &\STROUT{4-}{a,b,c}
    \end{align*}
  \end{itemize}
\end{frame}
% ----------------------------------------------------------------------
\begin{frame}[c]{Example}
  \begin{itemize}
  \item Consider the rules
    \begin{align*}
      &{\tt b.}\\
      &{\tt a\texttt{ :- } b,\texttt{ not }c.}
    \end{align*}
  \item <2-> This rule has one stable model
    \begin{align*}
      &\STROUT{3-}{}\\
      &\STROUT{3-}{c}\\
      &\STROUT{3-}{b}\\
      &\STROUT{4-}{b,c}\\
      &\STROUT{3-}{a}\\
      &\STROUT{3-}{a,c}\\
      &\{a,b\}\\
      &\STROUT{4-}{a,b,c}
    \end{align*}
  \end{itemize}
\end{frame}
% ----------------------------------------------------------------------
\begin{frame}[c]{Example}
  \begin{itemize}
  \item Consider the rules
    \begin{align*}
      &{\tt b.\quad   c.}\\
      &{\tt a\texttt{ :- } b,\texttt{ not }c.}
    \end{align*}
  \item <2-> This rule has one stable model
    \begin{align*}
      &\STROUT{3-}{}\\
      &\STROUT{3-}{c}\\
      &\STROUT{3-}{b}\\
      &\{b,c\}\\
      &\STROUT{3-}{a}\\
      &\STROUT{3-}{a,c}\\
      &\STROUT{3-}{a,b}\\
      &\STROUT{4-}{a,b,c}
    \end{align*}
  \end{itemize}
\end{frame}
% ----------------------------------------------------------------------
%
%%% Local Variables:
%%% mode: latex
%%% TeX-master: "../../main"
%%% End:
