% ----------------------------------------------------------------------
\begin{frame}[t]{Logic Programs as Propositional Formulas}
\bigskip
\alt<8->{
\(
{
\begin{array}{r@{}c@{}l@{\quad}l@{\quad}l@{\quad}l@{\quad}l@{\quad}}
\uncover<7>{P
& {} = {} &
\big\{
a\leftarrow \neg b
&
b\leftarrow \neg a
&
x\leftarrow a,\neg c
&
x\leftarrow y
&
y\leftarrow x,b
\big\}
\\[2mm]}
\makebox[2em]{\CF{P}\enspace}
& {} = {} &
\multicolumn{5}{l}{\mkern-8mu
\big\{a\leftrightarrow
    \big(
       \mbox{$\bigvee_{(a\leftarrow B)\in P}$}\BF{B}
    \big)
  \mid a\in\atom{P}
\big\}}
\\[1mm]
\makebox[2em]{\BF{B}\enspace}
& {} = {} &
\multicolumn{5}{l}{\mkern-8mu
      \mbox{$\bigwedge_{b\in B\cap\atom{P}}$}b \wedge
      \mbox{$\bigwedge_{\neg c\in B}$}\neg c
}
\\[1mm]
\makebox[2em]{\LF{P}\enspace}
& {} = {} &
\multicolumn{5}{l}{\mkern-8mu
  \big\{
  \big(
    \mbox{$\bigvee_{a \in L}$} a
  \big)
  \rightarrow
%  \neg
  \big(
    \mbox{$\bigvee_{a\in L,(a\leftarrow B)\in P,B\cap L=\emptyset}$}\BF{B}
  \big)
  \mid L \in \mathit{loop}(P)
  \big\}
}
\\
\end{array}
}
\)
}{
\(
{
\begin{array}{r@{}c@{}l@{\quad}l@{\quad}l@{\quad}l@{\quad}l@{\quad}}
P
& {} = {} &
\big\{
a\leftarrow \neg b
&
b\leftarrow \neg a
&
x\leftarrow \alert<5-6>{a},\neg c
&
\alert<5-6>{x}\leftarrow \alert<5-6>{y}
&
\alert<5-6>{y}\leftarrow \alert<5-6>{x},b
\big\}
\\[2mm]
\uncover<2->{
\makebox[2em]{\alt<2-3>{\RF{P}}{\CF{P}}\enspace}
& {} = {} &
\big\{
a\alt<4->{\leftrightarrow}{\leftarrow} \neg b
&
b\alt<4->{\leftrightarrow}{\leftarrow} \neg a
&
\multicolumn{2}{l}{\mkern-7mu x\alt<4->{\leftrightarrow}{\leftarrow} (a\wedge\neg c)\vee y}
&
y\alt<4->{\leftrightarrow}{\leftarrow} x\wedge b\big\}
\\[1mm]
}
\uncover<4->{
& {} \cup {} &
\big\{
c\leftrightarrow \bot
\big\}
}
\\[1mm]
\uncover<6->{
\makebox[2em]{\LF{P}\enspace}
& {} = {} &
\multicolumn{5}{l}{\mkern-7mu
\big\{
(x\vee y)\rightarrow a\wedge\neg c
\big\}}
\\[-2mm]
}
\end{array}
}
\)
}
\\\smallskip
\uncover<2->{
\uncover<1-7>{Classical models of \makebox[\widthof{$\CF{P}\cup\LF{P}$}][l]{$\alt<2-4>{\RF{P}\!:}{\alt<7->{\CF{P}\cup\LF{P}\!:}{\CF{P}\!:}}$}\only<2>{\hfill(only true atoms shown)}}
\begin{minipage}[b][10cm][t]{\textwidth}
\vspace*{1mm}
\alt<8->{
\begin{beamerboxesrounded}[upper=block title, lower=block body]{Theorem (Lin and Zhao)}
Let $P$ be a normal logic program and $X\subseteq\atom{P}$.

Then, $X$ is a stable model of $P$ iff $X\models\CF{P}\cup\LF{P}$.
\end{beamerboxesrounded}
\begin{itemize}
  \item Size of $\CF{P}$ is \alert{linear} in the size of $P$
  \item Size of $\LF{P}$ may be \alert{exponential} in the size of $P$~\nocite{lifraz04a}
\end{itemize}
}{
$\{b\}$,
\uncover<2-4>{$\{b,\alert<3-4>{c}\}$,}
\uncover<2-6>{$\{b,\alert<5-6>{x},\alert<5-6>{y}\}$,}
\uncover<2-4>{$\{b,\alert<3-4>{c},x,y\}$,
$\{a,\alert<3-4>{c}\}$,
$\{a,b,\alert<3-4>{c}\}$,}
$\{a,x\}$\uncover<2-4>{,
$\{a,\alert<3-4>{c},x\}$,
$\{a,x,\alert<3-4>{y}\}$,
$\{a,\alert<3-4>{c},x,y\}$,
$\{a,\alert<3-4>{b},x,y\}$,
$\{a,b,\alert<3-4>{c},x,y\}$}
\medskip
\begin{itemize}
 \item<3-> \alt<5->{\sout{Unsupported atoms}}{\alert<3,4>{Unsupported atoms}}
 \item<5-> \alt<7->{\sout{Unfounded atoms}}{\alert<5,6>{Unfounded atoms}}
\end{itemize}
}
\end{minipage}
}
\end{frame}
% ----------------------------------------------------------------------
% \begin{frame}[fragile]{Let's run it!}
% \footnotesize
% \pause
% \begin{semiverbatim}
% $ cat prg.lp
%
% \uncover<3->{\alert<3>{a :- not b.    b :- not a.    x :- a, not c.    x :- y.    y :- x, b.}}
% \end{semiverbatim}
% \pause[4]
% \begin{semiverbatim}
% $ clingo 0 prg.lp
% \uncover<5->{
% clingo version 4.5.0
% Reading from prg.lp
% Solving...
% Answer: 1
% \alert<5>{a x}
% Answer: 2
% \alert<5>{b}
% SATISFIABLE
%
% Models       : 2
% Calls        : 1
% Time         : 0.000s (Solving: 0.00s 1st Model: 0.00s Unsat: 0.00s)
% CPU Time     : 0.000s}
% \end{semiverbatim}
% \end{frame}
% ----------------------------------------------------------------------
% \begin{frame}{Genuine Stable Models Semantics}
%   \begin{itemize}
%   \item<2-> The reduct $\reduct{\phi}{X}$ of a formula $\phi$ relative to a set $X$ of atoms \pause[3]
%     \\ is defined as follows
%     \[
%     \begin{array}{ll}
%       \reduct{\phi}{X}=\bot &  \text{if $X\not\models \phi$}\\
%       \reduct{\phi}{X}=\phi &  \text{if  $\phi\in X$}\\
%       \reduct{\phi}{X}=(\reduct{\psi}{X} \circ \reduct{\mu}{X}) &
%                                                                   \text{if  $X\models \phi$ and  $\phi=(\psi\circ \mu)$ for $\circ\in\{\wedge,\vee,\rightarrow\}$}\\
%       \reduct{\phi}{X}=\top & \text{if $X\not\models \psi$ and $\phi=\neg \psi$}
%     \end{array}
%     \]
%   \end{itemize}
%   \uncover<4->{\vspace{-5pt}%
%   \begin{beamerboxesrounded}[upper=block title, lower=block body]{Definition (Gelfond and Lifschitz et al.)}
%     Let $\Phi$ be a formula and $X\subseteq\atom{\Phi}$.
%
%     Then, $X$ is a stable model of $\Phi$ if $X$ is a $\subseteq$-minimal model of~$\reduct{\Phi}{X}$
%   \end{beamerboxesrounded}}
%   \medskip
%   \begin{itemize}
%   \item<5-> \structure{Note} \
%     % \only<5>{\quad$a\vee\neg a$ is no tautology}
%     % \only<6>{$\neg a\vee\neg \neg a$ is a tautology}
%     % \only<7>
%     {$a$ and $\neg \neg a$ are not the same}
%   \end{itemize}
% \end{frame}
% ----------------------------------------------------------------------
%
%%% Local Variables:
%%% mode: latex
%%% TeX-master: "../../main"
%%% End:
