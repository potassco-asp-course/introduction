% ----------------------------------------------------------------------
\begin{frame}{An example}
  \pause\small
  \[
    \begin{array}{@{}r@{\,}c@{\,}l}
      P
      &=&
          \left\{
          \ r(a,b)\leftarrow, \ r(b,c)\leftarrow, \ t(X,Y) \leftarrow r(X,Y) \
          \right\}
      \\[2ex]\pause
      \mathcal{T}&=&\{a,b,c\}
      \\[1ex]\pause
      \mathcal{A}&=&\left\{
                     \begin{array}{@{\,}c@{\,}}
                       r(a,a),r(a,b),r(a,c), r(b,a),r(b,b),r(b,c), r(c,a),r(c,b),r(c,c),
                       \\
                       \!\! t(a,a),t(a,b),t(a,c), t(b,a),t(b,b),t(b,c), t(c,a),t(c,b),t(c,c)
                     \end{array}
      \right\}
    \end{array}
  \]
  \pause
  \[
    \ground{P}
    =
    \left\{
      \begin{array}{l@{\,}c@{\,}l@{\,}l@{\,}c@{\,}l@{\,}l@{\,}c@{\,}l@{\,}}
        r(a,b)&\leftarrow&, &&&\\
        r(b,c)&\leftarrow&, &&&\\
        \onslide<-5>{t(a,a)}&\onslide<-5>{\leftarrow}&\onslide<-5>{r(a,a),}&\onslide<-5>{t(b,a)}&\onslide<-5>{\leftarrow}&\onslide<-5>{r(b,a),}&\onslide<-5>{t(c,a)}&\onslide<-5>{\leftarrow}&\onslide<-5>{r(c,a),}\\
                     t(a,b) &             \leftarrow & \alt<7->{,}{r(a,b),}&\onslide<-5>{t(b,b)}&\onslide<-5>{\leftarrow}&\onslide<-5>{r(b,b),}&\onslide<-5>{t(c,b)}&\onslide<-5>{\leftarrow}&\onslide<-5>{r(c,b),}\\
        \onslide<-5>{t(a,c)}&\onslide<-5>{\leftarrow}&\onslide<-5>{r(a,c),}&             t(b,c) &             \leftarrow &\onslide<-6>{r(b,c),}&\onslide<-5>{t(c,c)}&\onslide<-5>{\leftarrow}&\onslide<-5>{r(c,c)}
      \end{array}
    \right\}
  \]

  \begin{itemize}\normalsize
  \item<8->[\itarrow] \ \alert{Grounding} aims at reducing the ground instantiation

    \ by applying semantic principles
  \end{itemize}

\end{frame}
% ----------------------------------------------------------------------
%
%%% Local Variables:
%%% mode: latex
%%% TeX-master: "../../main"
%%% End:
