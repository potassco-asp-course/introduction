% ----------------------------------------------------------------------
\begin{frame}{The logic of Here-and-There (HT)}
  \smallskip
  \begin{itemize}
  \item<2-> An \alert{interpretation} is a pair \tuple{H,T} of sets of atoms with $H \subseteq T$
    \begin{itemize}
    \item $H$ is called ``here'' and
    \item $T$ is called ``there''
    \end{itemize}
  \item<3-> \structure{Note} \ \tuple{H,T} is a simplified Kripke structure
    \medskip
  \item<4-> \structure{Intuition} \
    \begin{itemize}
    \item $H$ represents provably true atoms
    \item $T$ represents possibly true atoms
    \item atoms not in $T$ are false
    \end{itemize}
    \medskip
  \item<5-> \structure{Idea}
    \begin{itemize}
    \item $\tuple{H,T}\models\varphi\quad \sim\quad \varphi$ is provably true
    \item $\tuple{T,T}\models\varphi\quad \sim\quad \varphi$ is possibly true\pause[6] (ie, classically true)
    \end{itemize}
  \end{itemize}
\end{frame}
% ----------------------------------------------------------------------
\begin{frame}{Satisfaction}
  \begin{itemize}
  \item $\tuple{H,T} \models a$ if $a \in H$ \hfill for any atom $a$
  \item $\tuple{H,T} \models \varphi \wedge \psi$ if
    $\tuple{H,T} \models \varphi$
    and
    $\tuple{H,T} \models \psi$
  \item $\tuple{H,T} \models \varphi \vee \psi$ if
    $\tuple{H,T} \models \varphi$
    or
    $\tuple{H,T} \models \psi$

    \smallskip

  \item $\tuple{H,T} \models \varphi \rightarrow \psi$ if
    $\tuple{X,T} \models \varphi$ implies $\tuple{X,T} \models \psi$
    \\\qquad\qquad
    for both $X=H,T$

    \bigskip

  \item<2> \structure{Note} \ $\tuple{H,T} \models \neg\varphi$ if $\tuple{T,T} \not\models \varphi$
    \hfill since $\neg\varphi = \varphi\to\bot$

    \bigskip

  \item<3-> An interpretation \tuple{H,T} is a \alert{model} of $\varphi$, if $\tuple{H,T} \models \varphi$
  \end{itemize}
\end{frame}
% ----------------------------------------------------------------------
\begin{frame}{Classical tautologies}
\centering
  \[
    \begin{array}{| c | c || c | c | c | c | c |}
      \hline
      H         & T          & a      & \neg a & \alert{a \vee \neg a} & \neg\neg a & \alert{a \leftarrow \neg\neg a}
      \\\hline\hline
      \{ a \}   & \{ a \}    & \true  & \false & \true                 & \true      & \true
      \\\hline
      \emptyset & \{ a \}    & \false & \false & \alert{\false}        & \true      & \alert{\false}
      \\\hline
      \emptyset & \emptyset  & \false & \true  & \true                 & \false     & \true
      \\\hline
    \end{array}
  \]
\end{frame}
% ----------------------------------------------------------------------
\begin{frame}{Equilibrium models}
  \begin{itemize}
  \item A total interpretation $\tuple{T,T}$ is an \alert{equilibrium model}
    of\\ a formula $\varphi$,
    if
    \par
    \smallskip
    \begin{enumerate}\normalsize
    \item
      \(
      \tuple{T,T}\models\varphi
      \),
    \item
      \(
      \tuple{H,T}\not\models\varphi
      \)
      for all $H\subset T$
    \end{enumerate}
    \smallskip
  \item <2-> $T$ is called a \alert{stable model} of $\varphi$
    \bigskip
  \item<3-> \structure{Note} \ $\tuple{T,T}$ acts as a classical model
    \smallskip
  \item<4-> \structure{Note} \ $\tuple{H,T} \models P$ iff $H \models P^T$ \hfill ($P^T$ is the reduct of $P$ by $T$)
  \end{itemize}
\end{frame}
% ----------------------------------------------------------------------
%
%%% Local Variables:
%%% mode: latex
%%% TeX-master: "../../main"
%%% End:
